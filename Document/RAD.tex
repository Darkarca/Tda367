%
%  Title       :  Requirements and Analysis Document
%  Authors     :  Alex Gerdes
%  Created     :  August 20, 2018
%
%  Purpose     :  Tempate RAD for TDA367/DIT212
%
%-------------------------------------------------------------------------------

\documentclass[12pt,a4paper]{scrartcl}

\usepackage[hyphens]{url}
\usepackage{hyperref}
\usepackage{xcolor}
\usepackage{graphicx}
\usepackage{mathpazo}

\title{Requirements and Analysis Document for \ldots}
\author{authors}
\date{date\\version}

\begin{document}

\maketitle

\section{Introduction}

Background explaining why this application is needed (besides mandatory in
course). What's the problem addressed (use imagination)? What will it do? Who
will benefit/use from this application? In what situation will the application
be used? Define the application. General characteristics of application.

\subsection{Definitions, acronyms, and abbreviations}

Create word list to avoid confusion.


\section{Requirements}

\subsection{User Stories}

Use the template from the course website and list all user stories here. It is
fine to have them in an spreadsheet (or other application) at first, but they 
must end up here as well.

These user stories should describe what the user will be able to do. Write a 
the user stories in language of the customer, and give the a unique ID. List the
user stories in priority order.

\subsection{User interface}

Sketches, drawings and explanations of the application user interface (possible
navigation).


\section{Domain model}

Give a high level view overview of the application using a UML diagram.

\subsection{Class responsibilities}

Explanation of responsibilities of classes in diagram.


\section{References}


\end{document}
